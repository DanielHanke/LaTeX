\usepackage{tabularx}
\usepackage[pdftex]{graphicx}
\usepackage[spanish]{babel}
\usepackage[utf8]{inputenc}
\usepackage{array}
%\usepackage{hyperref}
\usepackage{listings}


%\usepackage{arimo}
%\usepackage{alegreya} %mas o menos%
%\usepackage{arev} %mejor%
\usepackage{bera}


%\usepackage[thinlines]{easytable}
%\usepackage[export]{adjustbox}
\usepackage{float}
\restylefloat{table}
\renewcommand{\arraystretch}{1.5}



\renewcommand*\familydefault{\sfdefault} %% Only if the base font of the document is to be sans serif
\usepackage[T1]{fontenc}
%\usepackage[table]{xcolor}% http://ctan.org/pkg/xcolor

\newcommand{\todo}[1]{\tiny{\emph{to-do: #1}}\normalsize}

\usepackage{amssymb}
%\usepackage{enumitem,xcolor}

\usepackage{tikz}
\usetikzlibrary{intersections}
\usepackage{epigraph}
\usepackage{lipsum}
\renewcommand\epigraphflush{flushleft}
\renewcommand\epigraphsize{\normalsize}
\setlength\epigraphwidth{0.7\textwidth}



% para trend!




% las negritas en bluetenaris
\let\oldtextbf\textbf
\renewcommand{\textbf}[1]{\textcolor{colorPrincipal}{\oldtextbf{#1}}}

% las italicas en bluetenaris
\let\oldemph\emph
\renewcommand{\emph}[1]{\textcolor{colorPrincipal}{\oldemph{#1}}}

\usepackage{sectsty}
\sectionfont{\color{colorPrincipal}}  % sets colour of chapters
\subsectionfont{\color{colorSecundario}}  % sets colour of sections
\subsubsectionfont{\color{colorTerciario}}  % sets colour of sections

\DeclareFixedFont{\titlefont}{T1}{ppl}{b}{it}{0.5in}
\makeatletter                       
\def\printauthor{%                  
    {\large \@author}}              
\makeatother

\newcommand\titlepagedecoration[1]{%
\begin{tikzpicture}[remember picture,overlay,shorten >= -10pt]
\coordinate (tp1) at ([yshift=2cm]current page.west);
\coordinate (tp2) at ([yshift=2cm,xshift=9cm]current page.west);
\coordinate (tp3) at ([yshift=-15pt,xshift=7cm]current page.north);
\coordinate (tp4) at ([yshift=-15pt]current page.north west);
% Place text to het its coordinates
\node[right] (titletext) at ([xshift=1cm,yshift=-5cm]current page.north west) {\parbox{\textwidth}{\color{white}#1}};

\path[name path=p1] ([xshift=-5cm]tp2) -- ([xshift=-5cm]tp3);
\path[name path=p2] (tp2) -- (tp3);
\path[name path=p3] (tp1 |- titletext.south) -- (titletext.south -| tp3);

\path[name intersections={of=p1 and p3,name=first}];
\path[name intersections={of=p2 and p3,name=second}];

\filldraw[colorTerciario!60!white] (first-1) -- (second-1) -- (tp3) -- ([xshift=-5cm]tp3) -- cycle;
\filldraw[colorTerciario!40!white] (tp4) -- ([xshift=-5cm]tp3) -- (first-1) -- (tp1 |- titletext.south) -- cycle;
\filldraw[colorTerciario!60!white] (tp1 |- titletext.south) -- (first-1) -- ([xshift=-5cm]tp2) -- (tp1) -- cycle;
\filldraw[colorTerciario!40!white] (first-1) -- (second-1) -- (tp2) -- ([xshift=-5cm]tp2) -- cycle;

% X = -6 <-> 16

\filldraw[colorPrincipal]  (-6,5) rectangle (2,4.5);
\filldraw[colorSecundario]  (2,5) rectangle (10,4.5);
\filldraw[colorTerciario]  (10,5) rectangle (18,4.5);

%\filldraw[gray!100!white]  (-3.+18,4.0    ) rectangle (-1.5+18,4.2); % 1w
%\filldraw[gray!80!white]    (-4.6+18,4.0	 ) rectangle (-3.1+18,4.2); %1.5w
%\filldraw[gray!60!white]    (-6.7+18,4.0	 ) rectangle (-4.7+18,4.2);
%\filldraw[gray!40!white]    (-9.3+18,4.0	 ) rectangle (-6.8+18,4.2);
%\filldraw[gray!20!white]    (-13.4+18,4.0) rectangle (-9.4+18,4.2);

%\filldraw[gray!100!white]  (-3.+18,4.0    ) rectangle (-1.5+18,4.2); % 1w
%\filldraw[gray!80!white]    (-4.6+18,4.0	 ) rectangle (-3.1+18,4.2); %1.5w
%\filldraw[gray!60!white]    (-6.7+18,4.0	 ) rectangle (-4.7+18,4.2);
%\filldraw[gray!40!white]    (-9.3+18,4.0	 ) rectangle (-6.8+18,4.2);
%\filldraw[gray!20!white]    (-13.4+18,4.0) rectangle (-9.4+18,4.2);



% Place text again, to have it on top
\node[right] (titletext) at ([xshift=1cm,yshift=-5cm]current page.north west) {\parbox{\textwidth}{\color{colorPrincipal}#1}};


\end{tikzpicture}%
}

\lstset{% general command to set parameter(s)
keywordstyle=\color{violet!55}\bfseries,
%keywordstyle=\color{blue}\bfseries\underbar,
% underlined bold black keywords
identifierstyle=\color{blue!60}, % nothing happens
commentstyle=\color{gray!70}, % white comments
stringstyle=\color{red!75}, % typewriter type for strings
numbers=left,
numberblanklines=false,
%frame=topline
backgroundcolor=\color{gray!10},
showstringspaces=false,
literate=*{0}{{\textcolor{violet}{0}}}{1}%
             {1}{{\textcolor{violet}{1}}}{1}%
             {2}{{\textcolor{violet}{2}}}{1}%
             {3}{{\textcolor{violet}{3}}}{1}%
             {4}{{\textcolor{violet}{4}}}{1}%
             {5}{{\textcolor{violet}{5}}}{1}%
             {6}{{\textcolor{violet}{6}}}{1}%
             {7}{{\textcolor{violet}{7}}}{1}%
             {8}{{\textcolor{violet}{8}}}{1}%
             {9}{{\textcolor{violet}{9}}}{1}%
             {.0}{{\textcolor{violet}{.0}}}{2}%
             {.1}{{\textcolor{violet}{.1}}}{2}%
             {.2}{{\textcolor{violet}{.2}}}{2}%
             {.3}{{\textcolor{violet}{.3}}}{2}%
             {.4}{{\textcolor{violet}{.4}}}{2}%
             {.5}{{\textcolor{violet}{.5}}}{2}%
             {.6}{{\textcolor{violet}{.6}}}{2}%
             {.7}{{\textcolor{violet}{.7}}}{2}%
             {.8}{{\textcolor{violet}{.8}}}{2}%
             {.9}{{\textcolor{violet}{.9}}}{2}%
             {=}{{\textcolor{violet}{=}}}{1}%
             {>}{{\textcolor{violet}{>}}}{1}%
             {<}{{\textcolor{violet}{<}}}{1}%
             {*}{{\textcolor{violet}{*}}}{1}%
             %{/}{{\textcolor{violet}{/}}}{1}% stuffs up comments
             {+}{{\textcolor{violet}{+}}}{1}%
             {-}{{\textcolor{violet}{-}}}{1}%
             {\%}{{\textcolor{violet}{\%}}}{1}%
             {:}{{\textcolor{violet}{:}}}{1}%
             {;}{{\textcolor{violet}{;}}}{1}%
             {,}{{\textcolor{violet}{,}}}{1}%
             {\&}{{\textcolor{violet}{\&}}}{1}%
             {(}{{\textcolor{violet}{(}}}{1}%
             {)}{{\textcolor{violet}{)}}}{1}%
             {\{}{{\textcolor{violet}{\{}}}{1}%
             {\}}{{\textcolor{violet}{\}}}}{1}%
             ,
} % no special string spaces

\renewcommand{\lstlistingname}{Ejemplo}% Listing -> Algorithm
\renewcommand{\lstlistlistingname}{Ejemplos de}