\documentclass[]{article}

\usepackage{enumitem,xcolor}
\usepackage[spanish]{babel}
\usepackage[utf8]{inputenc}
\usepackage{bera}
\renewcommand*\familydefault{\sfdefault} %% Only if the base font of the document is to be sans serif
\usepackage[T1]{fontenc}
\usepackage{xcolor,listings}

%opening
\title{}
\author{}

\begin{document}


\section{Relevamiento de Applet Uploader}

\par El applet trabaja con 3 entidades ( Ya Pablo me confirmó que deben ir las 3 entidades )

\begin{itemize}[label=\textcolor{blue}{\textbullet}]
	\item Elencos ( y toda su información asociada )
	\item Descartes
	\item Patrones

\end{itemize}

\newpage

\subsection{Elencos}

\par Para saber qué elencos debe transferir utiliza esta vista 
\emph{vwBMEUploaderNoDocumentados}

\par Items que debe retornar esa vista:

\begin{itemize}[label=\textcolor{blue}{\textbullet}]
	\item dbo.HistoricoElencos.IdElenco
	\item dbo.HistoricoElencos.NumeroElenco
	\item dbo.HistoricoElencos.TipoElenco
	\item dbo.HistoricoElencos.Expediente
	\item dbo.HistoricoElencos.Flags
	\item dbo.HistoricoElencos.FechaInicio
	\item dbo.HistoricoElencos.FechaFin
	\item dbo.HistoricoElencos.FechaEnvio
	\item dbo.HistoricoElencos.Mensaje
	\item dbo.HistoricoElencos.CodigoDespacho
	\item dbo.HistoricoElencos.SubCodigoDespacho
	\item dbo.HistoricoElencos.Descripcion
	\item dbo.HistoricoElencos.Semi
	\item dbo.HistoricoElencos.Darsena
	\item dbo.HistoricoElencos.Automatic
	\item dbo.HistoricoElencos.Logica
	\item dbo.HistoricoElencos.Contingencia
\end{itemize}

\par Con estas condiciones en la cláusula WHERE

\begin{itemize}[label=\textcolor{blue}{\textbullet}]
	\item dbo.HistoricoElencos.Flags = 1
	\item dbo.HistoricoElencos.FechaEnvio IS NOT NULL
\end{itemize}

\par A partir de esta vista se utiliza el  sp \emph{qryBMEUploaderTubosDelElenco} para acceder a cada tubo de este elenco pendiente
Esto trae toda la información de los tubos, el expediente, ciclo y demás ( antes se traía en varios sps, ahora éste hace todo ).


\par Items que debe retornar este sp:

\begin{itemize}[label=\textcolor{blue}{\textbullet}]
	\item HistoricoTubos.IdTubo,
	\item HistoricoTubos.[TimeStamp],
	\item HistoricoTubos.Fecha,
	\item HistoricoTubos.NroLote,
	\item HistoricoTubos.Status,
	\item HistoricoTubos.IdDescarte,
	\item HistoricoTubos.IdClave,
	\item HistoricoTubos.Secuencia,
	\item \begin{verbatim}dbo.fnBmeNumeroTrace(
			dbo.HistoricoClaves.ExpedienteDestino, 
			HistoricoTubos.NroTubo) AS NroTubo,\end{verbatim}
	\item HistoricoTubos.NroCupla,
	\item dbo.BMEDatosTubos.IdTubo
	\item dbo.BMEDatosTubos.NroTuboPintado
	\item dbo.BMEDatosTubos.LongitudPulsos
	\item dbo.BMEDatosTubos.PesoPulsos
	\item dbo.BMEDatosTubos.Longitud
	\item dbo.BMEDatosTubos.Peso
	\item dbo.BMEDatosTubos.Status
	\item dbo.BMEDatosTubos.IdElenco
	\item dbo.BMEDatosTubos.IdPaquete
	\item dbo.BMEDatosTubos.NroLoteExpe
	\item dbo.BMEDatosTubos.LongitudPintada
	\item dbo.BMEDatosTubos.PesoPintado
	\item dbo.BMEDatosTubos.PiesPintados
	\item dbo.BMEDatosTubos.LibrasPintadas
	\item dbo.BMEDatosTubos.CartelEstarcido
	\item dbo.BMEDatosTubos.CartelPunzonado
	\item dbo.BMEDatosTubos.LoteTT
	\item dbo.BMEDatosTubos.ColadaCupla
	\item dbo.BMEDatosTubos.Colada
	\item dbo.BMEDatosTubos.FechaEnvio
	\item dbo.BMEDatosTubos.Mensaje
	\item dbo.BMEDatosTubos.LoteCupla
	\item dbo.BMEDatosTubos.ColadaReal
	\item dbo.BMEDatosTubos.CicloCupla
	\item dbo.BMEDatosTubos.NroUnion
	\item dbo.BMEDatosTubos.NroSecu
	\item dbo.BMEDatosTubos.TraceFlag
	\item dbo.BMEDatosTubos.PesoTeorico
	\item dbo.BMEDatosTubos.PctVariacionPeso
	\item dbo.BMEDatosTubos.PesoMetrico
	\item dbo.BMEDatosTubos.IdLingada
	\item dbo.BMEDatosTubos.URC
	\item dbo.BMEDatosTubos.RfidData
	\item dbo.BMEDatosTubos.RfidTag
	\item dbo.BMEDatosTubos.CuplaVirtual

	\item dboHistoricoPaquetes.IdPaquete
	\item dboHistoricoPaquetes.IdElenco
	\item dboHistoricoPaquetes.NumeroPaquete
	\item dboHistoricoPaquetes.Flags
	\item dboHistoricoPaquetes.FechaInicio
	\item dboHistoricoPaquetes.FechaFin

	\item dbo.HistoricoElencos.IdElenco
	\item dbo.HistoricoElencos.NumeroElenco
	\item dbo.HistoricoElencos.TipoElenco
	\item dbo.HistoricoElencos.Expediente
	\item dbo.HistoricoElencos.Flags
	\item dbo.HistoricoElencos.FechaInicio
	\item dbo.HistoricoElencos.FechaFin
	\item dbo.HistoricoElencos.FechaEnvio
	\item dbo.HistoricoElencos.Mensaje
	\item dbo.HistoricoElencos.CodigoDespacho
	\item dbo.HistoricoElencos.SubCodigoDespacho
	\item dbo.HistoricoElencos.Descripcion
	\item dbo.HistoricoElencos.Semi
	\item dbo.HistoricoElencos.Darsena
	\item dbo.HistoricoElencos.Automatic
	\item dbo.HistoricoElencos.Logica
	\item dbo.HistoricoElencos.Contingencia

	\item dbo.HistoricoClaves.IdClave
	\item dbo.HistoricoClaves.IdEstacion
	\item dbo.HistoricoClaves.IdProducto
	\item dbo.HistoricoClaves.Status
	\item dbo.HistoricoClaves.FechaInicio
	\item dbo.HistoricoClaves.FechaFin
	\item dbo.HistoricoClaves.FechaSiderca
	\item dbo.HistoricoClaves.Turno
	\item dbo.HistoricoClaves.Escuadra
	\item dbo.HistoricoClaves.CicloOrigen
	\item dbo.HistoricoClaves.CicloDestino
	\item dbo.HistoricoClaves.ExpedienteOrigen
	\item dbo.HistoricoClaves.ExpedienteDestino
	\item dbo.HistoricoClaves.Colada
	\item dbo.HistoricoClaves.Campagna
	\item dbo.HistoricoClaves.Lote
	\item dbo.HistoricoClaves.ColadaReal
	\item dbo.HistoricoClaves.ColadaStock  
	\item BmedatosTubos.Colada AS ColadaTubo 
\end{itemize}

\par Con estas condiciones en la cláusula WHERE

\begin{itemize}[label=\textcolor{blue}{\textbullet}]
	\item BMEDatosTubos.IdElenco=@IdElenco 
	\item BMEDatosTubos.Status=1 
\end{itemize}

\par Con estas condiciones de order (ORDER BY)

\begin{itemize}[label=\textcolor{blue}{\textbullet}]
	\item NroTuboPintado	
\end{itemize}

\newpage

\subsection{Descartes}

\par Utiliza una vista para obtener los tubos descartados \emph{vwBMEUploaderDescartes}

\par Items que debe retornar esa vista:

\begin{itemize}[label=\textcolor{blue}{\textbullet}]
	\item dbo.HistoricoTubos.IdDescarte, 
	\item dbo.HistoricoTubos.Fecha, 
	\item dbo.HistoricoTubos.NroLote, 
	\item dbo.HistoricoTubos.NroTubo, 
	\item dbo.HistoricoTubos.NroCupla, 
	\item dbo.HistoricoTubos.Secuencia, 
	\item dbo.HistoricoDescartes.ID, 
	\item dbo.HistoricoDescartes.Codigo, 
	\item dbo.HistoricoDescartes.FechaDescarte,
	\item dbo.HistoricoDescartes.IDDefecto1,
	\item dbo.HistoricoDescartes.IDDefecto3,
	\item dbo.HistoricoDescartes.IDDefecto2, 
	\item dbo.HistoricoDescartes.IDDefecto4,
	\item dbo.HistoricoDescartes.IDDefecto5, 
	\item dbo.HistoricoDescartes.Obs, 
	\item dbo.HistoricoClaves.FechaSiderca, 
	\item dbo.HistoricoClaves.Turno, 
	\item dbo.HistoricoClaves.CicloOrigen,
	\item dbo.HistoricoClaves.CicloDestino,
	\item dbo.HistoricoClaves.ExpedienteOrigen,
	\item dbo.HistoricoClaves.ExpedienteDestino,
	\item dbo.HistoricoClaves.IdEstacion, 
	\item dbo.HistoricoClaves.Escuadra, 
	\item dbo.BMEDatosTubos.IdTubo, 
	\item ISNULL(dbo.BMEDatosTubos.NroTuboPintado, 0) as NroTuboPintado, 
	\item dbo.BMEDatosTubos.LongitudPulsos, 
	\item dbo.BMEDatosTubos.PesoPulsos, 
	\item ISNULL(dbo.BMEDatosTubos.Longitud , 0 ) as Longitud , 
	\item dbo.BMEDatosTubos.Status, 
	\item ISNULL(dbo.BMEDatosTubos.Peso, 0 ) as Peso , 
	\item dbo.BMEDatosTubos.IdElenco, 
	\item dbo.BMEDatosTubos.IdPaquete, 
	\item dbo.BMEDatosTubos.NroLoteExpe, 
	\item ISNULL(dbo.BMEDatosTubos.LongitudPintada, 0 ) as LongitudPintada, 
	\item ISNULL(dbo.BMEDatosTubos.PesoPintado, 0 )as PesoPintado , 
	\item ISNULL(dbo.BMEDatosTubos.PiesPintados , 0 ) as PiesPintados , 
	\item ISNULL(dbo.BMEDatosTubos.LibrasPintadas , 0 ) as LibrasPintadas , \item dbo.BMEDatosTubos.CartelEstarcido, 
	\item dbo.BMEDatosTubos.CartelPunzonado, 
	\item ISNULL( dbo.BMEDatosTubos.LoteTT, 0 ) as LoteTT , 
	\item dbo.BMEDatosTubos.ColadaCupla, dbo.BMEDatosTubos.FechaEnvio, 
	\item dbo.BMEDatosTubos.Mensaje, dbo.BMEDatosTubos.Colada, 
	\item dbo.BMEDatosTubos.LoteCupla, 
	\item dbo.BMEDatosTubos.ColadaReal, 
	\item dbo.BMEDatosTubos.CicloCupla, 
	\item dbo.BMEDatosTubos.NroUnion, 
	\item dbo.BMEDatosTubos.NroSecu, 
	\item dbo.BMEDatosTubos.TraceFlag
\end{itemize}

\par Con estas condiciones en la cláusula WHERE

\begin{itemize}[label=\textcolor{blue}{\textbullet}]
	\item dbo.HistoricoDescartes.Status = 1
	\item dbo.BMEDatosTubos.Status <> 1
	\item dbo.BMEDatosTubos.Status <> 5
	\item dbo.BMEDatosTubos.Status <> 11
	\item dbo.BMEDatosTubos.Status <> 12
	\item dbo.BMEDatosTubos.Status <> 98
\end{itemize}

\par Con estas condiciones de order (ORDER BY)

\begin{itemize}[label=\textcolor{blue}{\textbullet}]
	\item dbo.HistoricoTubos.Fecha DESC
\end{itemize}

\newpage

\subsection{Patrones}

\par Utiliza una vista para obtener los tubos descartados \emph{vwDocuRepPatrones}

\par Items que debe retornar esa vista:

\begin{itemize}[label=\textcolor{blue}{\textbullet}]
\item dbo.HistoricoTubos.IdTubo, 
\item dbo.Replicacion.ID, 
\item dbo.HistoricoTubosPatron.IdPatron, 
\item dbo.BMEDatosTubos.Longitud, 
\item dbo.BMEDatosTubos.Peso, 
\item dbo.HistoricoClaves.IdEstacion, 
\item dbo.HistoricoTubos.Fecha, 
\item dbo.BMEDatosTubos.LongitudPulsos, 
\item dbo.BMEDatosTubos.PesoPulsos,
\item \begin{verbatim}
	(SELECT     ValorFloat FROM          dbo.HistoricoPresetsPLC
	WHERE      (NroLote =
	(SELECT     MAX(NroLote) AS Expr1
		FROM          dbo.HistoricoPresetsPLC AS HistoricoPresetsPLC_9
		WHERE      (IdPreset = 10030) AND (NroLote <= ht.NroLote)))) AS CML1001,
	\end{verbatim}

\item \begin{verbatim}(SELECT     ValorFloat
FROM          dbo.HistoricoPresetsPLC AS HistoricoPresetsPLC_8
WHERE      (NroLote =
(SELECT     MAX(NroLote) AS Expr1
FROM          dbo.HistoricoPresetsPLC AS HistoricoPresetsPLC_7
WHERE      (IdPreset = 10028) AND (NroLote <= ht.NroLote)))) AS CAL1000,
\end{verbatim}
\item \begin{verbatim}(SELECT     ValorFloat
FROM          dbo.HistoricoPresetsPLC AS HistoricoPresetsPLC_6
WHERE      (NroLote =
(SELECT     MAX(NroLote) AS Expr1
FROM          dbo.HistoricoPresetsPLC AS HistoricoPresetsPLC_5
WHERE      (IdPreset = 10033) AND (NroLote <= ht.NroLote)))) AS CP1002,\end{verbatim}
\item \begin{verbatim}(SELECT     ValorFloat
FROM          dbo.HistoricoPresetsPLC AS HistoricoPresetsPLC_4
WHERE      (NroLote =
(SELECT     MAX(NroLote) AS Expr1
FROM          dbo.HistoricoPresetsPLC AS HistoricoPresetsPLC_3
WHERE      (IdPreset = 10032) AND (NroLote <= ht.NroLote)))) AS DES1003,\end{verbatim}
\item \begin{verbatim}(SELECT     ValorFloat
FROM          dbo.HistoricoPresetsPLC AS HistoricoPresetsPLC_2
WHERE      (NroLote =
(SELECT     MAX(NroLote) AS Expr1
FROM          dbo.HistoricoPresetsPLC AS HistoricoPresetsPLC_1
WHERE      (IdPreset = 10029) AND (NroLote <= ht.NroLote)))) AS CAC1004 \end{verbatim}
\end{itemize}

\par Con estas condiciones en la cláusula WHERE

\begin{itemize}[label=\textcolor{blue}{\textbullet}]
	\item dbo.BMEDatosTubos.Status = 5
	\item dbo.Replicacion.Procesado = 0
\end{itemize}

\subsection{Cambiar status de elenco}

\par Para cambiar el status de Elencos, el sistema utiliza el sp \emph{doBMETubosSetElencoFlags} que en este caso lo establece en \emph {3}


\end{document}

